\documentclass{exam}%echo por: lilberick
\usepackage{enumitem}
\usepackage{bigints}
\usepackage{draftwatermark}
\usepackage{fix-cm}
\title{UNIVERSIDAD NACIONAL DE INGENIERIA \\ DEPARTAMENTO DE MATEMATICAS \\ \large PRIMERA PRACTICA CALIFICADA DE MATEMATICAS III}
\author{LOS PROFESORES}
\date{Lima, 3 de Noviembre de 1973}
\begin{document}
\maketitle
\SetWatermarkLightness{ 0.9 }
\SetWatermarkText{Hecho por: LILBERICK}
\SetWatermarkScale{0.3}
\begin{enumerate}
	\item Dada la función $f$ definida por $f(x)=|x^2-2x-5|$ , $D_f=<-8,8>$, se pide:
		\begin{enumerate}[label=\alph*)]
			\item Determinar que restricciones se deben hacer sobre $D_f$ para que exista sobre cada restricción $f^*$
			\item Determinar en cada caso $f^*$
		\end{enumerate}
	\item Un helicóptero despega en un campo en un punto situado a 800 pies de un observador y se eleva verticalmente a razón de 25 pies/seg. Encontrar la razón de cambio con respecto al tiempo, del ángulo de elevación del observador con respecto al helicóptero, cuando este está situado a 600 pies encima del campo.
	\item Resolver
		\begin{enumerate}[label=\alph*)]
			\item $\bigintss\sqrt{\frac{\arccos{z}}{1-z^2}}\,dz$
			\item $\bigintsss 0^{x}2^{0^x}3^{0^x}\,dx$
			\item $\bigintsss\frac{ae^\theta+b}{ae^\theta-b}\,d\theta$
			\item $\bigintsss\frac{2+\ln{x}}{x}\,dx$
		\end{enumerate}
	\item Sea $f(u)=\arctan{\frac{u^2}{\sqrt{3}}}$. Determinar en qué intervalo de su dominio es cóncava hacia arriba.
	\item Determinar $f^\prime (x)$, si $f(x)=x^{a^a}+a^{x^a}+a^{a^x}$
\end{enumerate}
\section*{OBSERVACIONES}
\begin{enumerate}
	\item Escoger 4 de las 5 preguntas propuestas
	\item Sin tablas, libros, apuntes, etc
	\item Devolver esta hora sin anotaciones
\end{enumerate}
\section*{TIEMPO}		
	2 horas
\end{document}%echo por: lilberick

